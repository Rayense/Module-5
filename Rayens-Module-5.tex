\documentclass[]{article}
\usepackage{lmodern}
\usepackage{amssymb,amsmath}
\usepackage{ifxetex,ifluatex}
\usepackage{fixltx2e} % provides \textsubscript
\ifnum 0\ifxetex 1\fi\ifluatex 1\fi=0 % if pdftex
  \usepackage[T1]{fontenc}
  \usepackage[utf8]{inputenc}
\else % if luatex or xelatex
  \ifxetex
    \usepackage{mathspec}
  \else
    \usepackage{fontspec}
  \fi
  \defaultfontfeatures{Ligatures=TeX,Scale=MatchLowercase}
\fi
% use upquote if available, for straight quotes in verbatim environments
\IfFileExists{upquote.sty}{\usepackage{upquote}}{}
% use microtype if available
\IfFileExists{microtype.sty}{%
\usepackage{microtype}
\UseMicrotypeSet[protrusion]{basicmath} % disable protrusion for tt fonts
}{}
\usepackage[margin=1in]{geometry}
\usepackage{hyperref}
\hypersetup{unicode=true,
            pdftitle={Rayens Module 5},
            pdfborder={0 0 0},
            breaklinks=true}
\urlstyle{same}  % don't use monospace font for urls
\usepackage{color}
\usepackage{fancyvrb}
\newcommand{\VerbBar}{|}
\newcommand{\VERB}{\Verb[commandchars=\\\{\}]}
\DefineVerbatimEnvironment{Highlighting}{Verbatim}{commandchars=\\\{\}}
% Add ',fontsize=\small' for more characters per line
\usepackage{framed}
\definecolor{shadecolor}{RGB}{248,248,248}
\newenvironment{Shaded}{\begin{snugshade}}{\end{snugshade}}
\newcommand{\AlertTok}[1]{\textcolor[rgb]{0.94,0.16,0.16}{#1}}
\newcommand{\AnnotationTok}[1]{\textcolor[rgb]{0.56,0.35,0.01}{\textbf{\textit{#1}}}}
\newcommand{\AttributeTok}[1]{\textcolor[rgb]{0.77,0.63,0.00}{#1}}
\newcommand{\BaseNTok}[1]{\textcolor[rgb]{0.00,0.00,0.81}{#1}}
\newcommand{\BuiltInTok}[1]{#1}
\newcommand{\CharTok}[1]{\textcolor[rgb]{0.31,0.60,0.02}{#1}}
\newcommand{\CommentTok}[1]{\textcolor[rgb]{0.56,0.35,0.01}{\textit{#1}}}
\newcommand{\CommentVarTok}[1]{\textcolor[rgb]{0.56,0.35,0.01}{\textbf{\textit{#1}}}}
\newcommand{\ConstantTok}[1]{\textcolor[rgb]{0.00,0.00,0.00}{#1}}
\newcommand{\ControlFlowTok}[1]{\textcolor[rgb]{0.13,0.29,0.53}{\textbf{#1}}}
\newcommand{\DataTypeTok}[1]{\textcolor[rgb]{0.13,0.29,0.53}{#1}}
\newcommand{\DecValTok}[1]{\textcolor[rgb]{0.00,0.00,0.81}{#1}}
\newcommand{\DocumentationTok}[1]{\textcolor[rgb]{0.56,0.35,0.01}{\textbf{\textit{#1}}}}
\newcommand{\ErrorTok}[1]{\textcolor[rgb]{0.64,0.00,0.00}{\textbf{#1}}}
\newcommand{\ExtensionTok}[1]{#1}
\newcommand{\FloatTok}[1]{\textcolor[rgb]{0.00,0.00,0.81}{#1}}
\newcommand{\FunctionTok}[1]{\textcolor[rgb]{0.00,0.00,0.00}{#1}}
\newcommand{\ImportTok}[1]{#1}
\newcommand{\InformationTok}[1]{\textcolor[rgb]{0.56,0.35,0.01}{\textbf{\textit{#1}}}}
\newcommand{\KeywordTok}[1]{\textcolor[rgb]{0.13,0.29,0.53}{\textbf{#1}}}
\newcommand{\NormalTok}[1]{#1}
\newcommand{\OperatorTok}[1]{\textcolor[rgb]{0.81,0.36,0.00}{\textbf{#1}}}
\newcommand{\OtherTok}[1]{\textcolor[rgb]{0.56,0.35,0.01}{#1}}
\newcommand{\PreprocessorTok}[1]{\textcolor[rgb]{0.56,0.35,0.01}{\textit{#1}}}
\newcommand{\RegionMarkerTok}[1]{#1}
\newcommand{\SpecialCharTok}[1]{\textcolor[rgb]{0.00,0.00,0.00}{#1}}
\newcommand{\SpecialStringTok}[1]{\textcolor[rgb]{0.31,0.60,0.02}{#1}}
\newcommand{\StringTok}[1]{\textcolor[rgb]{0.31,0.60,0.02}{#1}}
\newcommand{\VariableTok}[1]{\textcolor[rgb]{0.00,0.00,0.00}{#1}}
\newcommand{\VerbatimStringTok}[1]{\textcolor[rgb]{0.31,0.60,0.02}{#1}}
\newcommand{\WarningTok}[1]{\textcolor[rgb]{0.56,0.35,0.01}{\textbf{\textit{#1}}}}
\usepackage{graphicx,grffile}
\makeatletter
\def\maxwidth{\ifdim\Gin@nat@width>\linewidth\linewidth\else\Gin@nat@width\fi}
\def\maxheight{\ifdim\Gin@nat@height>\textheight\textheight\else\Gin@nat@height\fi}
\makeatother
% Scale images if necessary, so that they will not overflow the page
% margins by default, and it is still possible to overwrite the defaults
% using explicit options in \includegraphics[width, height, ...]{}
\setkeys{Gin}{width=\maxwidth,height=\maxheight,keepaspectratio}
\IfFileExists{parskip.sty}{%
\usepackage{parskip}
}{% else
\setlength{\parindent}{0pt}
\setlength{\parskip}{6pt plus 2pt minus 1pt}
}
\setlength{\emergencystretch}{3em}  % prevent overfull lines
\providecommand{\tightlist}{%
  \setlength{\itemsep}{0pt}\setlength{\parskip}{0pt}}
\setcounter{secnumdepth}{0}
% Redefines (sub)paragraphs to behave more like sections
\ifx\paragraph\undefined\else
\let\oldparagraph\paragraph
\renewcommand{\paragraph}[1]{\oldparagraph{#1}\mbox{}}
\fi
\ifx\subparagraph\undefined\else
\let\oldsubparagraph\subparagraph
\renewcommand{\subparagraph}[1]{\oldsubparagraph{#1}\mbox{}}
\fi

%%% Use protect on footnotes to avoid problems with footnotes in titles
\let\rmarkdownfootnote\footnote%
\def\footnote{\protect\rmarkdownfootnote}

%%% Change title format to be more compact
\usepackage{titling}

% Create subtitle command for use in maketitle
\providecommand{\subtitle}[1]{
  \posttitle{
    \begin{center}\large#1\end{center}
    }
}

\setlength{\droptitle}{-2em}

  \title{Rayens Module 5}
    \pretitle{\vspace{\droptitle}\centering\huge}
  \posttitle{\par}
    \author{}
    \preauthor{}\postauthor{}
    \date{}
    \predate{}\postdate{}
  

\begin{document}
\maketitle

\hypertarget{project-management-for-reproducible-research}{%
\subsection{Project management for reproducible
research}\label{project-management-for-reproducible-research}}

This exercise is about practical stuff -- managing scientific projects
and workflows efficiently to avoid mistakes and keep the research moving
forward.

\hypertarget{learning-outcomes}{%
\subsection{Learning outcomes}\label{learning-outcomes}}

\begin{enumerate}
\def\labelenumi{\arabic{enumi}.}
\item
  Keeping file structure
\item
  Version control
\item
  Create a data analysis workflow, reporting, and the writing of
  scientific manuscripts
\end{enumerate}

\hypertarget{r-projects}{%
\subsection{R Projects}\label{r-projects}}

At this point, one might wonder how to keep track of all these files,
datasets, and figures. The answer: R projects. An R project consists of
a file in a project directory (with .Rproj extension) and the associated
data, scripts, and other files. When an R project is opened the
following occurs (from: \url{http://rstudio.com}):

\begin{itemize}
\tightlist
\item
  A new R session (process) is started
\item
  The .Rprofile file in the project's main directory (if any) is sourced
  by R
\item
  The .RData file in the project's main directory is loaded (if project
  options indicate that it should be loaded).
\item
  The .Rhistory file in the project's main directory is loaded into the
  RStudio History pane (and used for Console Up/Down arrow command
  history).
\item
  The current working directory is set to the project directory.
\item
  Previously edited source documents are restored into editor tabs
\item
  Other RStudio settings (e.g.~active tabs, splitter positions, etc.)
  are restored to where they were the last time the project was closed.
\end{itemize}

\hypertarget{version-control-git-and-github}{%
\subsection{Version control, Git, and
Github}\label{version-control-git-and-github}}

R projects are especially useful for \emph{version control}. Version
control is software for managing changes to code, data, and the other
files associated with a project. Version control facilitates undoing
revisions to an arbitrary point in the history of a file and the clean
synchronization of changes contributed by multiple users. \emph{Git} is
a popular version control system that works well with R/RStudio through
the Github repository service.

Create a Github account:

\begin{enumerate}
\def\labelenumi{\arabic{enumi}.}
\tightlist
\item
  Go to the Github website \url{http://github.com} and click on ``sign
  up''.
\item
  Create a profile (username, email, password).
\item
  Click ``start a project''.
\item
  Verify email address.
\item
  Create a project using the provided form.
\end{enumerate}

Link Github account to R/RStudio:

(There are lots of ways to link your installation of R with Github.
These instructions are for use with R/Rstudio installed on a Windows
computer such as in the teaching lab. Note that to use Github you also
have to have Git installed on your computer.)

\begin{enumerate}
\def\labelenumi{\arabic{enumi}.}
\tightlist
\item
  Open RStudio.
\item
  Select File -\textgreater{} New Project -\textgreater{} Version
  Control -\textgreater{} Git.
\item
  Go to the website for the project you created and click the green
  ``Clone or download'' button; copy the URL.
\item
  Paste the copied URL in the RStudio dialog box and navigate to the
  directory where you would like your project to be cloned. This will
  create a local copy of all the files associated with your project.
\end{enumerate}

You should now have a ``project'' open. In the file explorer you should
see the file \texttt{README.md} (if you created one), an \texttt{.Rproj}
file, and another file \texttt{.gitignore} (which, conveniently, you can
ignore). To include files (e.g.~data, scripts) in the project, simply
create or copy them into the directory. For instance:

\textbf{Exercise. Copy the MERS data file \texttt{cases.csv} and paste
it into your working directory.}

\textbf{Exercise. Create a new script following the prototype we
introduced. Your script should load the MERS data and make a plot.}

\begin{Shaded}
\begin{Highlighting}[]
\KeywordTok{library}\NormalTok{(ggplot2)}
\KeywordTok{library}\NormalTok{(lubridate)}
\end{Highlighting}
\end{Shaded}

\begin{verbatim}
## 
## Attaching package: 'lubridate'
\end{verbatim}

\begin{verbatim}
## The following object is masked from 'package:base':
## 
##     date
\end{verbatim}

\begin{Shaded}
\begin{Highlighting}[]
\KeywordTok{library}\NormalTok{(car)}
\end{Highlighting}
\end{Shaded}

\begin{verbatim}
## Loading required package: carData
\end{verbatim}

\begin{Shaded}
\begin{Highlighting}[]
\KeywordTok{library}\NormalTok{(ggthemes)}
\KeywordTok{library}\NormalTok{(plotly)}
\end{Highlighting}
\end{Shaded}

\begin{verbatim}
## Warning: package 'plotly' was built under R version 3.6.2
\end{verbatim}

\begin{verbatim}
## 
## Attaching package: 'plotly'
\end{verbatim}

\begin{verbatim}
## The following object is masked from 'package:ggplot2':
## 
##     last_plot
\end{verbatim}

\begin{verbatim}
## The following object is masked from 'package:stats':
## 
##     filter
\end{verbatim}

\begin{verbatim}
## The following object is masked from 'package:graphics':
## 
##     layout
\end{verbatim}

\begin{Shaded}
\begin{Highlighting}[]
\NormalTok{mers <-}\StringTok{ }\KeywordTok{read.csv}\NormalTok{(}\StringTok{'cases.csv'}\NormalTok{)}


\NormalTok{mers}\OperatorTok{$}\NormalTok{hospitalized[}\DecValTok{890}\NormalTok{] <-}\StringTok{ }\KeywordTok{c}\NormalTok{(}\StringTok{'2015-02-20'}\NormalTok{)}
\NormalTok{mers <-}\StringTok{ }\NormalTok{mers[}\OperatorTok{-}\DecValTok{471}\NormalTok{,]}
\NormalTok{mers}\OperatorTok{$}\NormalTok{onset2 <-}\StringTok{ }\KeywordTok{ymd}\NormalTok{(mers}\OperatorTok{$}\NormalTok{onset)}
\NormalTok{mers}\OperatorTok{$}\NormalTok{hospitalized2 <-}\StringTok{ }\KeywordTok{ymd}\NormalTok{(mers}\OperatorTok{$}\NormalTok{hospitalized)}
\end{Highlighting}
\end{Shaded}

\begin{verbatim}
## Warning: 5 failed to parse.
\end{verbatim}

\begin{Shaded}
\begin{Highlighting}[]
\KeywordTok{class}\NormalTok{(mers}\OperatorTok{$}\NormalTok{onset2)}
\end{Highlighting}
\end{Shaded}

\begin{verbatim}
## [1] "Date"
\end{verbatim}

\begin{Shaded}
\begin{Highlighting}[]
\NormalTok{day0 <-}\StringTok{ }\KeywordTok{min}\NormalTok{(}\KeywordTok{na.omit}\NormalTok{(mers}\OperatorTok{$}\NormalTok{onset2))}
\NormalTok{mers}\OperatorTok{$}\NormalTok{epi.day <-}\StringTok{ }\KeywordTok{as.numeric}\NormalTok{(mers}\OperatorTok{$}\NormalTok{onset2 }\OperatorTok{-}\StringTok{ }\NormalTok{day0)}

\KeywordTok{ggplot}\NormalTok{(}\DataTypeTok{data=}\NormalTok{mers) }\OperatorTok{+}\StringTok{ }
\StringTok{  }\KeywordTok{geom_bar}\NormalTok{(}\DataTypeTok{mapping=}\KeywordTok{aes}\NormalTok{(}\DataTypeTok{x=}\NormalTok{epi.day, }\DataTypeTok{fill=}\NormalTok{country)) }\OperatorTok{+}
\StringTok{  }\KeywordTok{labs}\NormalTok{(}\DataTypeTok{x=}\StringTok{'Epidemic Day'}\NormalTok{, }\DataTypeTok{y=}\StringTok{'Case Count'}\NormalTok{, }\DataTypeTok{title=}\StringTok{'Global count of MERS cases'}\NormalTok{,}
       \DataTypeTok{caption=}\StringTok{"Data from: https://github.com/rambaut/MERS-Cases/blob/gh-pages/data/cases.csv"}\NormalTok{)}
\end{Highlighting}
\end{Shaded}

\begin{verbatim}
## Warning: Removed 535 rows containing non-finite values (stat_count).
\end{verbatim}

\begin{verbatim}
## Warning: position_stack requires non-overlapping x intervals
\end{verbatim}

\includegraphics{Rayens-Module-5_files/figure-latex/unnamed-chunk-1-1.pdf}

Now you have two files in your local directory that you don't have on
the Github repository. To synchronize your local copy and the
respository, you need to \emph{Commit} the changes and then \emph{Push}
the changes to Github.

\textbf{Exercise. To \emph{commit} a change, navigate to the ``Git'' tab
in the top right explorer window. You will see a list of files in your
work directory. Select the files that need to be pushed to Github and
click on ``Commit''. A dialog box will open. In the top right there is
an editing window where you can register a comment describing the nature
of the commit.} \emph{I used GitKraken to push all changes to the Github
repository}

\textbf{Exercise. Once you have committed one or more changes (and
documented with comments), you need to \emph{push} the changes to the
archived version. To do this, click on ``Push''. You will need to enter
your Github credentials in the dialog box that pops up. Now refresh the
website for your project. The latest versions of your files should
appear.} \emph{I used GitKraken to push all changes to the Github
repository}

Github is very useful for collaboration. Multiple users can contribute
code to the same project. Projects can be ``branched'' and ``merged''
and Github provides tools to identify and resolve conflicts when
multiple programmers are working on the same project. Although it's fine
to use the Github repository as the definitive version and create a new
local copy (a new project) every time, you don't have to. Instead, you
can leave your local working directory as it is and simply \emph{pull}
your collaborators contributions every time you start a new session.

\hypertarget{statistically-literate-programming-with-r-markdown}{%
\subsection{Statistically literate programming with R
Markdown}\label{statistically-literate-programming-with-r-markdown}}

\emph{Literate programming} is a programming paradigm due to Linux
founder Donald Knuth in which natural langauge explanations of a
program's logic are interspersed with the code \emph{snippets} that
actually perform the computation. \emph{Statistically literate
programming} applies this paradigm to data analysis. Statistically
literate programming is the idea that the thought process of the data
analyst can be captured in a report that contains explanation and
interpretation, the code used to perform an analysis, and the products
of that analysis such as tables of data, quantities (e.g.~p-values) and
graphs.

There are a handful of ways that one can do statistically literate
programming with R/RStudio. In this exercise, we will use R/Markdown and
\texttt{knitr}. Markdown is a lightweight markup language that allows
documents to be rendered in html and other formats (e.g.~pdf) with a
minimum of special formatting. Knitr is a system for dynamic report
generation in R. Both should already be installed on your computer.

In general, a project developed with R/Markdown will consist of a
markdown document (with extension \texttt{.Rmd}) and the compiled
report. You should learn a little more about the functions of R/Markdown
and knitr, but first we will engage in a little learning-by-doing.

\textbf{Exercise. Create a basic R Markdown document. Go to File
-\textgreater{} New File -\textgreater{} R Markdown. Enter a title and
author into the dialog box. Select the desired default output format.
Save the resulting, automatically generated file. To compile the
document, click on ``Knit'' in the R Studio editor. Study the code and
the resulting report.}

You now know enough R programming that you should be able to discern
what is commentary and what is code in the \texttt{.Rmd} document.
First, notice that the \texttt{.Rmd} document begins with a header and
how it is set off from the rest of the document with three hyphens at
the start and end of the header. These are editable properties. Indeed,
there are many more options that we could encode in the header, but do
not need to bother with now.

Next, observe that \emph{R chunks} begin with three back ticks and curly
braces (with the argument ``r''). Everything within this section must be
properly formatted R code. R code chunks are indicated in the editor
window with a different background color. Notice that the third R code
chunk includes code to generate a plot. The compiled document shows this
code, but also the figure it produces! This is the beauty of
statistically literate programming: it keeps the explanation, code, and
results all together in a way that can be inspected and used both by the
data analyst and by others who come later. Moreover, oif the data change
(they get updated, there is a correction, the analyst decides to look at
a particular subset of the data), all that is required to generate a new
report is to recompile the document.

Finally, notice that natural language explanation is interspersed with
the code chunks. These chunks don't require any special designation
(like the hyphens for the header or the back ticks for the R code), but
are basically just ``everything else''. This natural language commentary
can nonetheless be ``marked up'' with special symbols that cause the
formatted document to display sections titles, bold face or italics,
clickable URLs, etc. In fact, the Markdown markup language enables quite
a lot of reasonably sophisticated typesetting.

In closing, we note that we have only scratched the surface of what can
be done with R/Markdown and knitr.

\textbf{Exercise. Visit the \href{http://rmarkdown.rstudio.com/}{R
Markdown website} and look around. Especially, read through the
\href{http://rmarkdown.rstudio.com/authoring_basics.html}{Authoring
Basics}.}

\textbf{Exercise. Borrowing from your earlier exercises, prepare an
analysis of the WNV or MERS data as a reproducible document using
R/Markdown. Compile to a pdf (if your desktop in class is not compiling
in PDF compile in HTML.}

\emph{In conducting a brief summary of the earlier analysis, we need to
begin by recoding to simplify the outcomes and calculating the
infectious period and coverting it into days.}

\begin{Shaded}
\begin{Highlighting}[]
\NormalTok{mers}\OperatorTok{$}\NormalTok{outcome2 <-}\StringTok{ }\KeywordTok{recode}\NormalTok{(mers}\OperatorTok{$}\NormalTok{outcome, }\StringTok{"''='Unknown'; 'fatal'='Fatal'; 'recovered'='Recovered'; '?recovered'='Suspect_Recovered'; 'recovered?'='Suspect_Recovered'; '?fatal'='Suspect_Fatal'; 'fatal?'='Suspect_Fatal'"}\NormalTok{)}

\NormalTok{mers}\OperatorTok{$}\NormalTok{infectious.period <-}\StringTok{ }\NormalTok{mers}\OperatorTok{$}\NormalTok{hospitalized2}\OperatorTok{-}\NormalTok{mers}\OperatorTok{$}\NormalTok{onset2 }\CommentTok{# calculate infectious period}
\KeywordTok{class}\NormalTok{(mers}\OperatorTok{$}\NormalTok{infectious.period) }\CommentTok{# these data are class "difftime"}
\end{Highlighting}
\end{Shaded}

\begin{verbatim}
## [1] "difftime"
\end{verbatim}

\begin{Shaded}
\begin{Highlighting}[]
\NormalTok{mers}\OperatorTok{$}\NormalTok{infectious.period <-}\StringTok{ }\KeywordTok{as.numeric}\NormalTok{(mers}\OperatorTok{$}\NormalTok{infectious.period, }\DataTypeTok{units =} \StringTok{"days"}\NormalTok{) }\CommentTok{# convert to days}
\NormalTok{mers}\OperatorTok{$}\NormalTok{infectious.period2 <-}\StringTok{ }\KeywordTok{ifelse}\NormalTok{(mers}\OperatorTok{$}\NormalTok{infectious.period}\OperatorTok{<}\DecValTok{0}\NormalTok{,}\DecValTok{0}\NormalTok{,mers}\OperatorTok{$}\NormalTok{infectious.period)}
\end{Highlighting}
\end{Shaded}

\emph{We can then plot some overview figures of the infectious period in
the MERS data}

\begin{Shaded}
\begin{Highlighting}[]
\KeywordTok{ggplot}\NormalTok{(}\DataTypeTok{data=}\NormalTok{mers, }\DataTypeTok{mapping=}\KeywordTok{aes}\NormalTok{(}\DataTypeTok{x=}\NormalTok{epi.day, }\DataTypeTok{y=}\NormalTok{infectious.period2)) }\OperatorTok{+}\StringTok{ }
\StringTok{  }\KeywordTok{geom_point}\NormalTok{(}\DataTypeTok{mapping =} \KeywordTok{aes}\NormalTok{(}\DataTypeTok{color=}\NormalTok{country)) }\OperatorTok{+}
\StringTok{  }\KeywordTok{facet_grid}\NormalTok{(outcome2 }\OperatorTok{~}\StringTok{ }\NormalTok{country) }\OperatorTok{+}\StringTok{ }
\StringTok{  }\KeywordTok{scale_y_continuous}\NormalTok{(}\DataTypeTok{limits =} \KeywordTok{c}\NormalTok{(}\DecValTok{0}\NormalTok{, }\DecValTok{50}\NormalTok{)) }\OperatorTok{+}
\StringTok{   }\KeywordTok{theme}\NormalTok{(}\DataTypeTok{axis.text.x =} \KeywordTok{element_text}\NormalTok{(}\DataTypeTok{angle =} \DecValTok{90}\NormalTok{))}\OperatorTok{+}
\StringTok{  }\KeywordTok{scale_x_continuous}\NormalTok{(}\DataTypeTok{breaks=}\KeywordTok{c}\NormalTok{(}\DecValTok{0}\NormalTok{, }\DecValTok{1000}\NormalTok{, }\DecValTok{2000}\NormalTok{))}\OperatorTok{+}
\StringTok{   }\KeywordTok{labs}\NormalTok{(}\DataTypeTok{x=}\StringTok{'Epidemic day'}\NormalTok{, }\DataTypeTok{y=}\StringTok{'Infectious period'}\NormalTok{,}
       \DataTypeTok{title=}\StringTok{'MERS Infectious Period'}\NormalTok{, }\DataTypeTok{caption=}\StringTok{"Data from: https://github.com/rambaut/MERS-Cases/blob/gh-pages/data/cases.csv"}\NormalTok{)}
\end{Highlighting}
\end{Shaded}

\begin{verbatim}
## Warning: Removed 728 rows containing missing values (geom_point).
\end{verbatim}

\includegraphics{Rayens-Module-5_files/figure-latex/unnamed-chunk-3-1.pdf}

\begin{Shaded}
\begin{Highlighting}[]
\KeywordTok{ggplot}\NormalTok{(}\DataTypeTok{data=}\NormalTok{mers, }\DataTypeTok{color=}\NormalTok{country) }\OperatorTok{+}\StringTok{ }
\StringTok{  }\KeywordTok{geom_violin}\NormalTok{(}\DataTypeTok{mapping =} \KeywordTok{aes}\NormalTok{(}\DataTypeTok{x=}\NormalTok{epi.day, }\DataTypeTok{y=}\NormalTok{outcome2, }\DataTypeTok{color=}\NormalTok{country)) }\OperatorTok{+}
\StringTok{   }\KeywordTok{labs}\NormalTok{(}\DataTypeTok{x=}\StringTok{'Epidemic day'}\NormalTok{, }\DataTypeTok{y=}\StringTok{'Outcome'}\NormalTok{,}
       \DataTypeTok{title=}\StringTok{'MERS Case Fatality Rate'}\NormalTok{, }\DataTypeTok{caption=}\StringTok{"Data from: https://github.com/rambaut/MERS-Cases/blob/gh-pages/data/cases.csv"}\NormalTok{)}
\end{Highlighting}
\end{Shaded}

\begin{verbatim}
## Warning: Removed 535 rows containing non-finite values (stat_ydensity).
\end{verbatim}

\begin{verbatim}
## Warning: position_dodge requires non-overlapping x intervals
\end{verbatim}

\includegraphics{Rayens-Module-5_files/figure-latex/unnamed-chunk-4-1.pdf}

\emph{We can simplify the violin plot of the infectious disease period
of the major players in the MERS outbreak}

\begin{Shaded}
\begin{Highlighting}[]
\KeywordTok{ggplot}\NormalTok{(}\DataTypeTok{data=}\NormalTok{mers, }\DataTypeTok{mapping=}\KeywordTok{aes}\NormalTok{(}\DataTypeTok{x=}\NormalTok{epi.day, }\DataTypeTok{y=}\NormalTok{infectious.period2)) }\OperatorTok{+}\StringTok{ }
\StringTok{  }\KeywordTok{geom_violin}\NormalTok{(}\DataTypeTok{mapping =} \KeywordTok{aes}\NormalTok{(}\DataTypeTok{color=}\NormalTok{country)) }\OperatorTok{+}
\StringTok{  }\KeywordTok{scale_y_continuous}\NormalTok{(}\DataTypeTok{limits =} \KeywordTok{c}\NormalTok{(}\DecValTok{0}\NormalTok{, }\DecValTok{50}\NormalTok{)) }\OperatorTok{+}
\StringTok{   }\KeywordTok{labs}\NormalTok{(}\DataTypeTok{x=}\StringTok{'Epidemic day'}\NormalTok{, }\DataTypeTok{y=}\StringTok{'Infectious period'}\NormalTok{, }\DataTypeTok{title=}\StringTok{'MERS infectious period (positive values only) over time'}\NormalTok{, }\DataTypeTok{caption=}\StringTok{"Data from: https://github.com/rambaut/MERS-Cases/blob/gh-pages/data/cases.csv"}\NormalTok{)  }\OperatorTok{+}
\StringTok{  }\KeywordTok{scale_color_fivethirtyeight}\NormalTok{()}\OperatorTok{+}\StringTok{ }
\StringTok{  }\KeywordTok{theme_fivethirtyeight}\NormalTok{()}
\end{Highlighting}
\end{Shaded}

\begin{verbatim}
## Warning: Removed 728 rows containing non-finite values (stat_ydensity).
\end{verbatim}

\begin{verbatim}
## Warning: position_dodge requires non-overlapping x intervals
\end{verbatim}

\begin{verbatim}
## Warning: This manual palette can handle a maximum of 3 values. You have
## supplied 8.
\end{verbatim}

\includegraphics{Rayens-Module-5_files/figure-latex/unnamed-chunk-5-1.pdf}

\textbf{Exercise. Add an interactive plotly graph to your reproducible
document. Compile to HTML.} \emph{This is an interactive plot of the
MERS Case Fatality Rate}

\begin{Shaded}
\begin{Highlighting}[]
\NormalTok{finalthing <-}\StringTok{ }\KeywordTok{ggplot}\NormalTok{(}\DataTypeTok{data=}\NormalTok{mers, }\DataTypeTok{color=}\NormalTok{country) }\OperatorTok{+}\StringTok{ }
\StringTok{  }\KeywordTok{geom_violin}\NormalTok{(}\DataTypeTok{mapping =} \KeywordTok{aes}\NormalTok{(}\DataTypeTok{x=}\NormalTok{epi.day, }\DataTypeTok{y=}\NormalTok{outcome2, }\DataTypeTok{color=}\NormalTok{country)) }\OperatorTok{+}
\StringTok{   }\KeywordTok{labs}\NormalTok{(}\DataTypeTok{x=}\StringTok{'Epidemic day'}\NormalTok{, }\DataTypeTok{y=}\StringTok{'Outcome'}\NormalTok{,}
       \DataTypeTok{title=}\StringTok{'MERS Case Fatality Rate'}\NormalTok{, }\DataTypeTok{caption=}\StringTok{"Data from: https://github.com/rambaut/MERS-Cases/blob/gh-pages/data/cases.csv"}\NormalTok{)}
\KeywordTok{ggplotly}\NormalTok{(finalthing)}
\end{Highlighting}
\end{Shaded}

\begin{verbatim}
## Warning: Removed 535 rows containing non-finite values (stat_ydensity).
\end{verbatim}

\begin{verbatim}
## Warning: position_dodge requires non-overlapping x intervals
\end{verbatim}

\includegraphics{Rayens-Module-5_files/figure-latex/unnamed-chunk-6-1.pdf}

This document -- and all the others used for this workshop -- were
written in R/Markdown.


\end{document}
